% Copyright (c) 2008-2009 solvethis
% Copyright (c) 2010-2017,2021 Casper Ti. Vector
% Copyright (c) 2021 Kurapica
% All rights reserved.
%
% Redistribution and use in source and binary forms, with or without
% modification, are permitted provided that the following conditions are
% met:
%
% * Redistributions of source code must retain the above copyright notice,
%   this list of conditions and the following disclaimer.
% * Redistributions in binary form must reproduce the above copyright
%   notice, this list of conditions and the following disclaimer in the
%   documentation and/or other materials provided with the distribution.
% * Neither the name of Peking University nor the names of its contributors
%   may be used to endorse or promote products derived from this software
%   without specific prior written permission.
%
% THIS SOFTWARE IS PROVIDED BY THE COPYRIGHT HOLDERS AND CONTRIBUTORS "AS
% IS" AND ANY EXPRESS OR IMPLIED WARRANTIES, INCLUDING, BUT NOT LIMITED TO,
% THE IMPLIED WARRANTIES OF MERCHANTABILITY AND FITNESS FOR A PARTICULAR
% PURPOSE ARE DISCLAIMED. IN NO EVENT SHALL THE COPYRIGHT HOLDER OR
% CONTRIBUTORS BE LIABLE FOR ANY DIRECT, INDIRECT, INCIDENTAL, SPECIAL,
% EXEMPLARY, OR CONSEQUENTIAL DAMAGES (INCLUDING, BUT NOT LIMITED TO,
% PROCUREMENT OF SUBSTITUTE GOODS OR SERVICES; LOSS OF USE, DATA, OR
% PROFITS; OR BUSINESS INTERRUPTION) HOWEVER CAUSED AND ON ANY THEORY OF
% LIABILITY, WHETHER IN CONTRACT, STRICT LIABILITY, OR TORT (INCLUDING
% NEGLIGENCE OR OTHERWISE) ARISING IN ANY WAY OUT OF THE USE OF THIS
% SOFTWARE, EVEN IF ADVISED OF THE POSSIBILITY OF SUCH DAMAGE.

{
	\ctexset{section = {
		format+ = {\centering}, beforeskip = {40bp}, afterskip = {15bp}
	}}
	\specialchap{北京大学学位论文原创性声明和使用授权说明}

	% 学校书面要求本页面不要页码,但在给出的 Word 模版中又有页码。
	% 此处以学校书面要求为准。
	\thispagestyle{empty}
	
	% 替换扫描pdf,去除includegraphics前注释
	\begin{textblock}{1}(0,0)
		\colorbox{white}{
			% \includegraphics[height = 1.2\textheight]{lwsm_180xxxxxxx.pdf}
		}
	\end{textblock}

	\mbox{}\vspace*{-3em}
	\section*{原创性声明}

	本人郑重声明:
	所呈交的学位论文,是本人在导师的指导下,独立进行研究工作所取得的成果。
	除文中已经注明引用的内容外,
	本论文不含任何其他个人或集体已经发表或撰写过的作品或成果。
	对本文的研究做出重要贡献的个人和集体,均已在文中以明确方式标明。
	本声明的法律结果由本人承担。
	\vskip 1em
	\rightline{%
		论文作者签名:\hspace{5em}%
		日期:\hspace{2em}年\hspace{2em}月\hspace{2em}日%
	}

	\section*{%
		学位论文使用授权说明\\[-0.33em]
		\textmd{\zihao{5}(必须装订在提交学校图书馆的印刷本)}%
	}

	本人完全了解北京大学关于收集、保存、使用学位论文的规定,即:
	\begin{itemize}
		\item 按照学校要求提交学位论文的印刷本和电子版本;
		\item 学校有权保存学位论文的印刷本和电子版,
			并提供目录检索与阅览服务,在校园网上提供服务;
		\item 学校可以采用影印、缩印、数字化或其它复制手段保存论文;
		\item 因某种特殊原因须要延迟发布学位论文电子版,
			授权学校在 $\Box$\nobreakspace{}一年 /
			$\Box$\nobreakspace{}两年 /
			$\Box$\nobreakspace{}三年以后在校园网上全文发布。
	\end{itemize}
	\centerline{(保密论文在解密后遵守此规定)}
	\vskip 1em
	\rightline{%
		论文作者签名:\hspace{5em}导师签名:\hspace{8em}%
	}
	\vskip 0.5em
	\rightline{%
		日期:\hspace{2em}年\hspace{2em}月\hspace{2em}日\hspace{10em}%
	}

    % 不建议方式,签字请使用门户版本!
	% 若须排版二维码,请将二维码图片重命名为“barcode”,
    % 二维码内容为: 北京大学 xx学院 xx专业 xxx
    % 容错设置25%
    % 转为合适的图片格式,并放在当前目录下,然后去掉下面 2 行的注释。
    % \vfill\noindent
    % \includegraphics[height = 5em]{barcode}
}

% vim:ts=4:sw=4